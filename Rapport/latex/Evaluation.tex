\documentclass[../master.tex]{subfile}
\begin{document}

\section{Evaluation}
\subsection{How we designed and implemented the tests}
Throughout the project we have not tested the DIKUArcade code, since this already come with an extensive test project itself.
\\[10pt]
We test the loading process of a level file to the concrete implementations of each object, piece wise.
First by testing that the Loader under the assumption that it is given a valid filepath to a uncorrupted level file load the file and returns a list of string where every lines are none empty and has the expected output. Then we use a mock up implementation of IFetcher to test the LevelParser, this makes it easier to test since we know the expected output. By doing so we test the whole process of the loading of a file.
\\[10pt]
We test the Movement patterns with mock up dynamic shapes to test that they move as expected. This makes each moving entity in the game does not move unintendedly. e.i. the customer does not move out over the platform and so on.
\\[10pt]
For all SpaceTaxi Entities we have only tested for none DIKUArcade.Entity methods and attributes, so as collision. Since most of the collidwith methods alter the players behaviour we need to have a player represent to test this too.
\\[10pt]
The Level class we have not tested, this is by the assumption that since we has test all subentities of this class and only use pretested entities and methods we know as long it gets a valid IParser implementation it should work as intended.
\\[10pt]
For the Timer class we have done an extensive amount of white box testing of each methods. Since the AddMilliSecond call AddSecond and this call AddMinutes and this calls AddHours with the number of seconds, minutes and hours represented as milliseconds, seconds or minutes. we know that if AddMilliSecond adds the right amount of seconds and so on all the way we don't need to test for adding hours as milliseconds, but only every step once. We furthermore test that after an update of the timer the timer counters has changed accordingly, and if the times hits zero it should stop and we test the pause method to check that it pause and that it does not keep subtracting time while paused.

\end{document}