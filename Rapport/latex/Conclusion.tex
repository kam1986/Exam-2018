\section{Conclusion}
As shown by our completed project, we have created a functioning playable game using the DIKUArcade game engine, that can be compiled and run on multiple platforms. The game is as well tested for the purpose to test if the functionalities we have implemented works as intended. Our version of the game \textit{Space Taxi} has very limited features, but it is almost complete in the sense that it meets alomst all the given specifications. Sadly we could not figure out how to make so the player can pickup cusomters. However, we still believe we have met all requirements, (except the feature where you can pickup customers) for this project and have succeeded all of our goals we have sat for ourselves (see \hyperref[sec:Goals]{Goals}).\\
 
The game for example has a level-builder that currently creates 2 ASCII levels, where the player can control a taxi which is being affected by physics, can teleport to the next level by colliding with the portal in the top middle of the screen. The player can land on platforms where the customer is standing (has no other functions), the ships explodes by colliding with an obstacle and it explodes by landing to fast. We have made so the game contains a statemachine which gives the player the opportunity to start a new game or quit the game in a main menu. As well as pause, quit or start a new game in this state.\\

In addition to that we have learned more about the programming language C\# and its practical application, where in this particular situation, how it can be used to create a computer game. We have also learned to use different design patterns and formed a new way to think about the design of a program. That was, we had to use object oriented analysis and list up different design concepts in 3 responsibility tables. That helped us, to get an overview over which concepts we wanted to use, what their association should be and what they should be a part of. We used those new techniques to develop our game, and learned new ways of thinking and designing of programs, on the way. While developing we have been getting more routined using Github, but it have caused a lot of problems under the way, but not as bad so it have destroyed anything that could keep us away from finishing our work. In this deliverable we have refactored and made the overall structure of the code more readable, and by that more maintainble, which helps understandment for further development by us or other peoples. However, given more time we could have done more refactored, but as our solution is now, we are satisfied enough with what we have.\\